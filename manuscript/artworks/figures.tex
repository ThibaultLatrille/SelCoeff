\documentclass[8pt]{beamer}

\usepackage{amssymb,amsfonts,amsmath,amsthm,mathtools}
\usepackage{lmodern}
\usepackage{xfrac, nicefrac, bm}
\usepackage{pgfplots, pgf,tikz}
\usepackage{enumitem}
\usepgfplotslibrary{fillbetween}
\usebackgroundtemplate{\tikz\node[opacity=0]{};}
\setbeamertemplate{footline}[frame number]{}
\setbeamertemplate{navigation symbols}{}
\setbeamertemplate{footline}{}
\usefonttheme{serif}
\pgfplotsset{compat=1.16}

\definecolor{RED}{HTML}{EB6231}
\definecolor{YELLOW}{HTML}{E29D26}
\definecolor{BLUE}{HTML}{5D80B4}
\definecolor{LIGHTGREEN}{HTML}{6ABD9B}
\definecolor{GREEN}{HTML}{8FB03E}
\definecolor{PURPLE}{HTML}{BE1E2D}
\definecolor{BROWN}{HTML}{A97C50}
\definecolor{PINK}{HTML}{DA1C5C}

\pgfplotsset{every axis/.append style={line width=1pt}}
\pgfplotscreateplotcyclelist{colors}{LIGHTGREEN\\YELLOW\\RED\\GREEN\\BLUE\\}


\newcommand{\der}{\mathrm{d}}
\newcommand{\e}{\text{e}}
\newcommand{\Ne}{N_{\text{e}}}
\newcommand{\pnps}{\pn / \ps}
\newcommand{\proba}{\mathbb{P}}
\newcommand{\pfix}{\proba_{\text{fix}}}

\newcommand{\ci}{{\color{BLUE}{\textbf{ATT}}}}
\newcommand{\cj}{{\color{YELLOW}\textbf{ATG}}}
\newcommand{\nuci}{{\color{BLUE}\textbf{T}}}
\newcommand{\nucj}{{\color{YELLOW}\textbf{G}}}
\newcommand{\aai}{{\color{BLUE}\textbf{Ile}}}
\newcommand{\aaj}{{\color{YELLOW}\textbf{Met}}}
\newcommand{\Fi}{{F_{\aai}}}
\newcommand{\Fj}{{F_{\aaj}}}
\newcommand{\aaitoj}{{\aai \rightarrow \aaj}}
\newcommand{\nucitoj}{\nuci \rightarrow \nucj}
\newcommand{\citoj}{\ci \rightarrow \cj}
\newcommand{\AtoB}{A \rightarrow B}
\newcommand{\itoj}{ i \rightarrow j }

\newcommand{\dn}{d_N}
\newcommand{\ds}{d_S}
\newcommand{\dnds}{\dn / \ds}
\newcommand{\Sphy}{S_{0}}
\newcommand{\Sphyclass}{\mathcal{C}}
\newcommand{\SphyMean}{\overline{\Sphy}}
\newcommand{\divStrongDel}{\Sphy < -3}
\newcommand{\divDel}{-3 < \Sphy < -1}
\newcommand{\divWeakDel}{-1 < \Sphy < 0}
\newcommand{\divWeakAdv}{0 < \Sphy < 1}
\newcommand{\divAdv}{ \Sphy > 1}
\newcommand{\PdivStrongDel}{\proba \left[ \divStrongDel \right]}
\newcommand{\PdivDel}{\proba \left[ \divDel \right]}
\newcommand{\PdivWeakDel}{\proba \left[ \divWeakDel \right]}
\newcommand{\PdivWeakAdv}{\proba \left[ \divWeakAdv \right]}
\newcommand{\PdivAdv}{\proba \left[ \divAdv \right]}
\newcommand{\given}{\mid}
\newcommand{\Spop}{S}
\newcommand{\SpopMean}{\overline{\Spop}}
\newcommand{\polyDel}{\Spop < -1}
\newcommand{\polyNeutral}{-1 < \Spop < 1}
\newcommand{\polyAdv}{ \Spop > 1}
\newcommand{\PpolyDel}{\proba \left[ \polyDel \right]}
\newcommand{\PpolyNeutral}{\proba \left[ \polyNeutral \right]}
\newcommand{\PpolyAdv}{\proba \left[ \polyAdv \right]}

\begin{document}
	\begin{frame}
		\begin{itemize}[label=$\bullet$]
			\item $s_{\AtoB}$ is the selection coefficient of allele $B$ in a population of $A$.
			\item $w_A$ is the Wrightian fitness (survival $\times$ fecundity) of allele $A$.
			\item $f_A = \ln (w_A) $ is the Malthusian fitness of allele $A$.
		\end{itemize}
		\begin{align*}
			s_{\AtoB} &  = \frac{w_B - w_A}{w_A} \\
							     &  = \frac{w_B}{w_A} - 1 \\
								 &  \simeq \ln \left( \frac{w_B}{w_A} \right) \\
								 &  = \ln (w_B) - \ln (w_A) \\
								 &  = f_B - f_A. \\
		\end{align*}
		$s_{\AtoB}$ = 0.01 means that B is favored with a increased chance of survival or fecundity by 1\%.
	\end{frame}
	\begin{frame}
		For a diploid population of size $\Ne$ the initial frequency of a mutation with selection coefficient $s$ is $p = 1/2\Ne$. Its probability of fixation
denoted $\pfix(s)$ is given by:
		\begin{align*}
		\pfix(s) &  = \frac{1 - \e^{ -2s}}{1 - \e^{ -4 \Ne s}}  \\
		 	     & \simeq \frac{2s}{1 - \e^{ -4 \Ne s}}.
		\end{align*}
		The special case of a neutral allele can be obtained by taking the limit $s \rightarrow 0$:
		\begin{align*}
			\pfix(0) &  = \lim_{s \rightarrow 0} \pfix(s), \\
					 &  = \lim_{s \rightarrow 0} \frac{2s}{1 - \e^{ -4 \Ne s}} \\
					 &  = \lim_{s \rightarrow 0} \frac{2s}{1 - (1 - 4 \Ne s)} \\
					 &  = \lim_{s \rightarrow 0} \frac{2s}{4 \Ne s} \\
					 & = \frac{1}{2 \Ne}.
		\end{align*}
	\end{frame}
	\begin{frame}
		\begin{itemize}[label=$\bullet$]
			\item $\Ne$ is the number of diploid individuals.
			\item $q$ is the substitution rate of new alleles.
			\item $\mu$ is the mutation rate of new alleles.
			\item $\pfix(s)$ is the probability of fixation of new allele, with selection coefficient $s$.
		\end{itemize}
		\begin{align*}
			q &  = 2 \Ne \times \mu \times \pfix(s), \\
			&  = 2 \Ne \times \mu \times \frac{2s}{1 - \e^{ -4 \Ne s}}, \\
			&  =   \mu \times \frac{S}{1 - \e^{ - S}} \text{ with } S=4 \Ne s.
		\end{align*}
	\end{frame}
	\begin{frame}
		\centering
		\begin{tikzpicture}
			\begin{axis}[
				width=\textwidth,
				height=0.5\textwidth,
				ylabel={$\frac{Q}{\mu} = \frac{S}{1 - \e^{-S}}$},
				xlabel={Scaled selection coefficient ($S=4 \Ne s$)},
				cycle list name=colors,
				domain=-10:10,
				ymin=0.0, ymax=5.0,
				samples=200,
				legend entries={$\frac{S}{1 - \e^{-S}}$},
				legend cell align=left,
				minor tick num=2,
				axis x line=bottom,
				axis y line=left,
				legend style={at={(0.02,0.9)},anchor=north west}
				]
				\addplot[line width=2.0pt, BLUE]{ x / (1 - exp(- x))};
				\addplot[name path=B, dashed, YELLOW, line width=0.5pt] coordinates {(-1, 0) (-1, 10)};
				\addplot[name path=A, line width=0pt] coordinates {(-10, 0) (-10, 10)};
				\addplot[black, dashed, line width=1.0pt]{1.0};
				\addplot[black, dashed, line width=1.0pt] coordinates {(0, 0) (0, 10)};
				\addplot[name path=C, dashed, YELLOW, line width=0.5pt] coordinates {(1, 0) (1, 10)};
				\addplot[name path=D, line width=0pt] coordinates {(10, 0) (10, 10)};
				\addplot[fill=RED, opacity=0.2] fill between[ of = A and B];
				\addplot[fill=YELLOW, opacity=0.2] fill between[ of = B and C];
				\addplot[fill=GREEN, opacity=0.2] fill between[ of = C and D];
			\end{axis}
		\end{tikzpicture}
	\end{frame}
	\begin{frame}
		\begin{itemize}[label=$\bullet$]
			\item $q_{\citoj}$ is the substitution rate from codon $\ci$ to $\cj$.
			\item $\mu_{\nucitoj}$ is the mutation rate from nucleotide $\nuci$ to $\nucj$.
			\item $S_{\aaitoj}=4\Ne s_{\aaitoj}$ is the scaled selection coefficient from $\aai$ to $\aaj$.
			\item $\Fi=4\Ne f_{\aai} $ is the scaled fitness of Isoleucine.
			\item $\Fj=4\Ne f_{\aaj} $ is the scaled fitness of Methionine.
		\end{itemize}
		\begin{align*}
				q_{\citoj}  &  = \mu_{\nucitoj}   \dfrac{S_{\aaitoj}}{1 - \e^{-S_{\aaitoj}} } \\
				&  = \mu_{\nucitoj}   \dfrac{\Fj - \Fi}{1 - \e^{\Fi - \Fj} } \\
		\end{align*}
	\end{frame}
	\begin{frame}
	\begin{itemize}[label=$\bullet$]
		\item $q_{\itoj}$ is the substitution rate from codon $i$ to $j$.
		\item $\mu_{\itoj}$ is the mutation rate from codon $i$ to $j$.
		\item $ F_i $ is the scaled fitness of codon $i$ ($F_j$ for codon $j$).
	\end{itemize}
	\begin{equation*}
		\begin{dcases}
			q_{\itoj} = \mu_{\itoj} & \text{if synonymous,} \\
			q_{\itoj} = \mu_{\itoj} \frac{F_j - F_i}{1 - \e^{ F_i - F_j}} & \text{if non-synonymous.}\\
		\end{dcases}
	\end{equation*}
	\end{frame}
	\begin{frame}
		Before fixation or extinction, the probability of an allele to be at a certain frequency can be related to its selection coefficient ($s$) and to the effective population size ($\Ne$).
		\begin{itemize}[label=$\bullet$]
			\item $g(x) \der x $ is the expected time for which the population frequency of derived allele is in the range $(x, x+\der x)$ before eventual absorption
		\end{itemize}
		\begin{align*}
			g(x, S) & = \dfrac{\left( 1 - \e^{- 2 s }\right) \left( 1 - \e^{-4 \Ne s(1-x)}\right)}{ s (1 - \e^{-4 \Ne s})x(1-x)} \\
			        & \approx \dfrac{2 \left( 1 - \e^{-S(1-x)}\right)}{(1 - \e^{-S})x(1-x)}
		\end{align*}
	\end{frame}
	\begin{frame}
		\centering
		\begin{tikzpicture}[
		declare function={
			f(\x,\k)= 2 * (1 - exp(-\k * (1-\x))) / ((1 - exp(- \k))*\x*(1-\x));
		},]
		\begin{axis}[
		width=\textwidth,
		height=0.6\textwidth,
		ylabel={$g(x, S)$},
		xlabel={frequency of the derived allele ($x$)},
		domain=0.05:0.95,
		cycle list name=colors,
		samples=200,
		legend entries={$S=12$, $ S=4$, $S=0$, $S=-4$, $S=-12$},
		legend cell align=left,
		minor tick num=2,
		axis x line=bottom,
		axis y line=left,
		legend style={at={(0.3,0.7)},anchor=north west}
		]
		\addplot{f(\x, 12)};
		\addplot{f(\x, 4)};
		\addplot{2 / x};
		\addplot{f(\x, -4)};
		\addplot{f(\x, -12)};
		\end{axis}
		\end{tikzpicture}
	\end{frame}
	\begin{frame}
		\begin{itemize}[label=$\bullet$]
			\item $n$ is the number of sampled chromosomes (twice the number of individuals).
			\item $1 \leq i \leq n - 1 $ is number of chromosomes with the derived allele.
			\item $\beta_i$ is the number of SNPs with derived allele carried by $i$ chromosomes.
			\item $\beta_i' = \beta_i \times i $ is the scaled $\beta_i$.
		\end{itemize}
	\end{frame}
	\begin{frame}
		\begin{align*}
			H & = \frac{2}{n(n-1)} \left( \sum_{i=1}^{n-1} \beta_i \times (n-i) \times i   - \sum_{i=1}^{n-1} \beta_i \times i^2 \right)
		\end{align*}
		\begin{itemize}[label=$\bullet$]
			\item $H = 0$:  no evidence of deviation from neutrality. 
			\item $H > 0$: deficit of SNPs at high frequency (negative selection).
			\item $H < 0$: excess of SNPs at high frequency (positive selection).
		\end{itemize}
	\end{frame}
	\begin{frame}
		\begin{align*}
			\divStrongDel \\
			\divDel \\
			\divWeakDel \\
			\divWeakAdv \\
			\divAdv \\
			{\color{RED}\bm{\divStrongDel}} \\
			{\color{YELLOW}\bm{\divDel}} \\
			{\color{LIGHTGREEN}\bm{\divWeakDel}} \\
			{\color{GREEN}\bm{\divWeakAdv}} \\
			{\color{BLUE}\bm{\divAdv}} \\
			\PpolyDel \\
			\PpolyNeutral \\
			\PpolyAdv \\
			\proba \left[ \divAdv \given \polyAdv \right] 
		\end{align*}
	\end{frame}
	\begin{frame}
		\begin{align*}
			\proba \left[ \Sphy \given \Spop \right] \\
			\proba \left[ \Spop \given \Sphy \right] \\
			\proba \left[ \Sphy \right] \\
			\proba \left[ \Spop \right] \\
			\proba \left[ \Sphy \given \Spop \right] = \frac{ \proba \left[ \Spop \given \Sphy \right] \proba \left[ \Sphy \right] }{\proba \left[ \Spop \right]} \\
			\proba \left[ {\color{BLUE}\bm{\divAdv}} \given \polyAdv \right] \\
			\proba \left[ \polyAdv \given {\color{BLUE}\bm{\divAdv}} \right] \\
			\proba \left[ {\color{BLUE}\bm{\divAdv}} \right] \\
			\proba \left[  \polyAdv \right] \\
			\proba \left[ {\color{BLUE}\bm{\divAdv}} \given \polyAdv \right] = \frac{ \proba \left[ \polyAdv \given {\color{BLUE}\bm{\divAdv}} \right] \proba \left[ {\color{BLUE}\bm{\divAdv}} \right] }{\proba \left[  \polyAdv \right]}
		\end{align*}
	\end{frame}
\end{document}